{%%%%%%%%%%%%%%%%%%%%%%% ШАПКА %%%%%%%%%%%%%%%%%%%%%%%%%%%%
\documentclass[twoside,a4paper]{msmb} % Класс документа
\usepackage{longtable}
\usepackage{lipsum}
\usepackage{eurosym}
\usepackage{rotating}

%%%%%%%%%%%%%%%%%%%% Параметры страницы %%%%%%%%%%%%%%%%%%%
\evensidemargin 5mm \oddsidemargin 5mm \voffset -10mm \textheight 230mm \textwidth 150mm
%%%%%%%%%%%%%%%%%%%%%%%%%%%%%%%%%%%%%%%%%%%%%%%%%%%%%%%%%
\begin{document}
\selectlanguage{russian}
\yearpub{2011}                                      % Год издания. Заполняет редакция.
\num{24}                                            % Номер журнала. Заполняет редакция.
\renewcommand{\filename}{akorovyansky-mblock-1}                  % Вставьте имя вашего файла.
\author{А.А.~Коровянский} % Автор или группа авторов.
\shrauthor{А.А.~Коровянский}          % Автор или группа авторов еще раз.
\email{korovyansk@gmail.com}  % Электронный адрес; если адреса нет, то строку закомментировать.
\title{Обзор решений для реализации прототипа замка мобильной биометрической системы доступа M-Block} % Название статьи.
\shrtitle{Обзор решений для реализации прототипа замка M-Block\ldots}                 % Часть названия статьи для колонтитулов.
\udc{000.000}                                       % Вставьте УДК.
\org{ Омский государственный университет им. Ф.М.Достоевского}   % Организация.
%\thanks{}                                          % Раскомментируйте, если необходимо.
\annotation{В статье рассматриваются существующие средства для реализации прототипа замка мобильной биометрической системы доступа M-Block. В результате сравнительного анализа определяется наиболее подходящее решение.} % это аннотация статьи на русском языке. Постарайтесь сделать развернутую аннотацию.
\keywords{обзор решений, биометрия, мобильные устройства, биометрическая система доступа, инновационность, ОмГУ, прототип, M-Block, Raspberry Pi, Android, Arduino, AllWinner A1X}

%%%%%%%%%%%%%%%%%%%%%%%%%%%%%%%%%%%%%%%%%%%%%%%%%%%%%%%%%%%%%%%%
%%%%%%%%%%%% для реферирования зарубежом  %%%%%%%%%%%%%%%%%%%%%%
%%%%%%%%%%%%%%%%%%%%%%%%%%%%%%%%%%%%%%%%%%%%%%%%%%%%%%%%%%%%%%%% 
% Эти данные в самой статье не появляются, но используются для 
% формирования специальных страниц журнала, реферирования и регистрации в elibrary.ru
\authorEng{A.A.~Korovyansky} % Автор или группа авторов.
\titleEng{} % Название статьи.
\orgEng{}   % Организация.
\annotationEng{} % это аннотация статьи на ангнлийском языке
\keywordsEng{}


\maketitle % Формирование оглавления.


%%%%%%%%%%%%%%%%%%%%%% Т Е К С Т %%%%%%%%%%%%%%%%%%%%%%%%%
\section*{Введение} % Звездочка для того, чтобы раздел не нумеровался
При реализации любого программно-аппаратного комплекса большое значение имеет обзор, сравнительный анализ и выбор наиболее подходящего средства для реализации. Мотивацией написания данной статьи является решение данной задачи для прототипа замка мобильной биометрической системы доступа M-Block, разрабатываемой на факультете компьютерных наук ОмГУ им. Ф.М. Достоевского.

Биометрическая система доступа (БСД) --- это программно-аппаратный комплекс, решающий задачу разграничения физического доступа к целевому объекту путем проверки биометрических данных. Таким образом, для открытия дверей используются специальные замки и биометрические алгоритмы, верифицирующие человека по отпечатку пальца, внешности, голосу или другому биометрическому фактору.

В основе M-Block лежит идея использования возможностей смартфонов и других современных мобильных устройств для создания инновационной и промышленно значимой мобильной БСД. Система должна обеспечивать высокий уровень безопасности, поддерживать различные факторы биометрической авторизации и иметь возможность централизованной и простой настройки политик доступа.

На рис 1. представлена принципиальная схема M-Block. Три ключевых компонента системы: мобильное приложение, приложение замка и приложение сервера. 

\begin{figure}[ht] %
\centering
\includegraphics[width=\textwidth]{mblock_scheme.eps}\\
% Указывается размер рисунка (каким Вы его хотите видеть)% и имя вставляемого файла
\caption{Схема M-Block}% Подпись рисунка должна быть обязательно
\label{pic}% Метка для ссылки на рисунок.
\end{figure}

Последовательность событий при взаимодействии пользователя с системой:
\begin{enumerate}
\item Пользователь подходит к двери с целью открыть ее.
\item Мобильное приложение M-Block устанавливает защищенное соединение с приложением замка M-Block.
\item Приложение замка передает мобильному приложению необходимую информацию для верификации (например, какие биометрические данные нужно получить от пользователя).
\item Пользователь с помощью мобильного приложения формирует необходимые биометрические данные для верификации (например, для голосовой биометрии --- цифровую запись определенной фразы).
\item Мобильное приложение передает сформированные данные замку.
\item Приложение замка устанавливает защищенное соединение с сервером, передает ему биометрические данные и другую необходимую информацию.
\item Сервер M-Block запускает необходимый биометрический алгоритм, верифицирует и авторизует пользователя.
\item Сервер сообщает приложению замка результаты верификации и авторизации.
\item В случае положительного ответа приложение замка открывает дверь, в противном случае пользователю передается информация о причинах отказа.
\end{enumerate}

Как и любая другая БСД, M-Block является весьма сложным программно-аппаратном комплексом. Более того,  требования к ней могут меняться в процессе разработки, тестирования и апробации. Поэтому в первую очередь разработка системы должна начинаться  с реализации рабочего прототипа, к которому предъявляется жесткое условие возможности добавления и изменения требований в процессе разработки.

Целью данной статьи является обзор и выбор наиболее подходящего средства для реализации прототипа аппаратной и программной части замка.

\section{Критерии оценки}
Для определения лучшего средства решения поставленной задачи необходимо сформулировать критерии оценки. Важно отметить, что перечисленные ниже критерии имеют значимость для реализации прототипа и могут быть неподходящими при выборе средств для реализации промышленной версии.

\subsection{Взаимодействия с электро-механическим устройствами}
Для того, чтобы открывать и закрывать дверь, замок M-Block должен взаимодействовать с промышленным электро-механическим дверным замком. Поэтому устройство должно обязательно предоставлять программно-аппаратные возможности для управления электро-механическим приводом.

\subsection{Скорость и удобство разработки}
Для прототипа большое значение имеет скорость и удобство разработки, поэтому устройство должно поддерживать программирование на высокоуровневом языке программирования с использованием удобной и эффективной IDE и других инструментов разработки.

\subsection{Вычислительные ресурсы}
Приложения замка в ходе своей работы должно обмениваться сообщениями с мобильным приложением и сервером M-Block, обрабатывать и передавать значительные объемы биометрических данных. Поэтому для прототипа замка ставятся следующие требования: частота процессора - не менее 128 МГц, размер оперативной памяти - не менее 100 МБ, размер внутренней памяти (или подключаемой) - не менее 100 МБ.

\subsection{Беспроводное взаимодействие}
Замок должен устанавливать беспроводное защищенное соединение с мобильным приложением M-Block, поэтому прототип должен поддерживать как минимум один из современных способов беспроводной передачи данных на небольших расстояниях - Bluetooth или WiFi.

\subsection{Взаимодействие по локальной сети}
Замок должен устанавливать защищенное соединение с сервером M-Block, поэтому прототип должен обеспечивать поддержку популярных стандартов локальных сетей Ethernet или WiFi.

\subsection{Стек TCP/IP}
Исходя из поставленной задачи прототип замка должен обеспечивать работу со стеком протоколов TCP/IP и предоставлять возможности реализации безопасного протокола на основе TCP и UDP сокетов. Также в качестве транспорта для протокола взаимодействия между замком и сервером может быть выбран HTTP, поэтому крайне желательна поддержка HTTP/HTTPS при программировании устройства замка.

\subsection{Многозадачность}
Прототип замка должен одновременно коммуницировать с мобильным приложением и сервером M-Block, поэтому решение для прототипа должно предоставлять средства для организации многозадачности.

\subsection{Документация разработчика и поддержка}
Качество документации, наличие сторонних расширений, количество статей и различных учебных материалов в сети Интернет, активность коммьюнити разработчиков для каждого конкретного решения --- все это является очень важным критерием для прототипа, так как минимизирует время на решение вопросов, возникающих в процессе разработки.

\subsection{Стоимость}
Для прототипа замка стоимость не является ключевым фактором, так как прототип является уникальной и единичной реализацией. Однако, данный параметр должен рассматриваться как дополнительный и значимый, в случае равенства в сравнении решений по вышеперечисленным критериям.

\section{Существующие решения}
На данный момент существуют различные средства решения сформулированной задачи. В ходе подготовительной работы автором статьи было выделено подмножество предпочтительных решений, то есть заведомо более подходящих для поставленной задачи среди всего спектра решений. В их число вошли решения на базе Arduino, AllWinner A1X, Android и Raspberry Pi. В рамках данной статьи будут детально рассмотрены все предпочтительные решения и в результате сравнительного анализа по сформулированным критериям будет выявлено наиболее подходящее.

\subsection{Arduino}
Расширяемая аппаратно-вычислительная платформа, основными компонентами которой являются платы ввода/вывода, подключаемые шилды и среда разработки на языке Processing. Arduino предлагает различные модели плат, но все они основаны на микроконтроллере Atmel AVR, имея небольшие вычислительную мощность и размер по памяти\cite{Arduino}.

\bgroup
\def\arraystretch{1.5}%  1 is the default, change whatever you need

\begin{longtable}{| p{.2\textwidth} | p{.6\textwidth} | c |} 

\hline
Критерий & Комментарий & Оценка \\
\hline
Взаимодействие с электро-механическими  устройствами &

Arduino поддерживает прямое программирование портов GPIO, что является одной из его самых сильных сторон\cite{ArduinoGPIO}. &

+ \\

\hline
Скорость и удобство разработки &

Язык Processing, а также IDE для программирования Arduino являются достаточно хорошими и проверенными решениями, но значительно проигрывают в скорости и удобстве разработки современным высокоуровневыми языкам программирования, и соответствующим им IDE.

Также при использовании большого количества дополнительных шилдов может возникнуть серьезная проблема нехватки GPIO разъемов на плате и необходимости взаимоисключения шилдов. &

+/$-$\\

\hline
Вычислительные ресурсы &

Решения на базе Arduino имеют очень ограниченную и явно недостаточную для прототипа вычислительную мощность. Частота микроконтроллера не превышает 16 МГц, а размер оперативой памяти - 8 КБ\cite{Arduino Memory}. &

$-$\\

\hline
Беспроводное взаимодействие &

Arduino позволяет реализовать беспроводное взаимодействие по стандартам Bluetooth и WiFi за счет использования специальных плат или подключения дополнительных соответствующих шилдов. Так, например, взаимодействие по Bluetooth возможно за счет использование платы Arduino BT{ArduinoBT} или подключения шилда XBee\cite{XBee}. &

+\\

\hline
Взаимодействие по локальной сети &

Путем подключения дополнительных шилдов Ethernet\cite{ArduinoEthernet} и WiFi\cite{ArduinoWiFi} возможна реализация взаимодействия по соответствующему стандарту локальной сети. При этом плата Arduino может выступать как в роли клиента\cite{ArduinoClient}, так и сервера\cite{ArduinoServer}.&

+\\

\hline
Стек TCP/IP &

Стандартные библиотеки Arduino характеризуются очень низкоуровневым подходом к реализации взаимодействия по протоколам стека TCP/IP и отсутствием поддержки некоторых протоколов или определенных их возможностей. Например, Arduino не поддерживает взаимодействие по протоколу HTTPS по причине недостаточной вычислительной мощности. &

$-$\\

\hline
Многозадачность &

Arduino не имеет встроенных средств для обеспечения многозадачности. Поддержка многозадачности возможна только за счет использования специальных сторонних библиотек или RTOS\cite{RTOSArduino}. Однако, оба варианта приведут к значительному усложнению реализации. В итоге, многозадачность в Arduino является весьма проблемным вопросом с большим количеством <<подводных камней>>. &

$-$\\

\hline 
Документация разработчика и поддержка & 

Официальный сайт Arduino предоставляет большое количество обучающих материалов и хорошую документацию разработчика. Также вокруг платформы существует большое коммьюнити активных разработчиков, что значительно облегчает процесс решения сложных вопросов, возникающих в процессе разработки. &

+\\

\hline

Стоимость &

\EUR{80} \cite{ArduinoBuy} &

\\

\hline
\caption{Arduino} % needs to go inside longtable environment
\end{longtable}
\egroup

\subsection{Raspberry Pi}
Одноплатный компьютер размером с кредитную карту, имеющий аппаратные характеристики, достаточные для комфортной работы современных десктопных и мобильных операционных систем. За счет наличия 26 портов GPIO и сигнального процессора Raspberry Pi представляет хорошие возможности для разработки на его базе различных программно-аппаратных решений\cite{RaspberryPi}.

\bgroup
\def\arraystretch{1.5}%  1 is the default, change whatever you need

\begin{longtable}{| p{.2\textwidth} | p{.6\textwidth} | c |} 

\hline
Критерий & Комментарий & Оценка \\
\hline
Взаимодействие с электро-механическими  устройствами &

Raspberry Pi имеет 26 портов GPIO и поддерживает их прямое программирование\cite{RPIHardware}. Также стоит отметить факт существования проекта raspberry-gpio-python, позволяющего управлять GPIO на высокоуровневом языке Python\cite{RPiGPIOPython}. &

+\\

\hline
Скорость и удобство разработки &

Официально поддерживаемым языком для разработки приложений на Raspberry Pi является Python\cite{RPiFAQ}, для которого существует большое количество различных IDE. Оба эти фактора обеспечивают хорошие возможности для скоростной и комфортной разработки приложений под Raspberry Pi. &

+\\

\hline
Вычислительные ресурсы &

Raspberry обладает достаточными физическими характеристиками, имея 700 МГц процессор, 256 или 512 МБ памяти \cite{RPiFAQ} и возможность подключения до 32 ГБ внешней памяти. &

+\\

\hline
Беспроводное взаимодействие &

Реализация взаимодействия по Bluetooth и WiFi возможна путем подключения к USB порту Raspberry Pi WiFi/Bluetooth адаптера. &

+\\

\hline
Взаимодействие по локальной сети &

Raspberry Pi предоставляет встроенные средства для поддержки стандарта Ethernet и за счет расширения позволяет обеспечить поддержку WiFi. При этом Raspberry Pi может выступать как в роли клиента, так и в роли сервера.&

+\\

\hline
Стек TCP/IP &

Стандартной операционной системой Raspberry Pi является Raspbian\cite{Raspbian} --- оптимизированная и преднастроенная версия Debian. Raspbian содержит в себе полноценную реализацию стека TCP/IP, предоставляя необходимые средства для создания TCP и UDP сокетов, а также взаимодействию по протоколам HTTP и HTTPS. &

+\\

\hline
Многозадачность &

Raspbian, также как и Debian, является многозадачной операционной системой. &

+\\

\hline 
Документация разработчика и поддержка & 

Для Raspberry Pi в сети Интернет можно найти большое количество документации и различных учебных материалов. Также вокруг Raspberry Pi сформировалось большое и очень активное коммьюнити разработчиков, которое активно развивает платформу, каждый день появляются новые статьи о решении самых различных задач с помощью Raspberry Pi. Все это позволяет очень быстро находить ответ на различные вопросы, возникающие в процессе разработки для Raspberry Pi. &

+\\

\hline

Стоимость &

\EUR{21.60}\cite{RPiBuy}&

\\

\hline
\caption{Raspberry Pi} % needs to go inside longtable environment
\end{longtable}
\egroup

\subsection{Android}
Открытая мобильная операционная система, основанная на ядре Linux. На данный момент на рынке представлено огромное количество различных устройств под управлением Android, в том числе смартфонов, планшетов, электронных книг, домашних мульти-медиа центров, фотоаппаратов и телевизоров\cite{Android}.

Для решения сформулированной задачи наиболее удачным форм-фактором является смартфон, так как при прочих равных он имеет компактный размер, необходимые аппаратные возможности и наиболее качественную документацию. Поэтому оценка Android будет даваться именно на основе возможностей Android смартфонов.

\bgroup
\def\arraystretch{1.5}%  1 is the default, change whatever you need

\begin{longtable}{| p{.2\textwidth} | p{.6\textwidth} | c |} 

\hline
Критерий & Комментарий & Оценка \\
\hline
Взаимодействие с электро-механическими  устройствами &

Существующие Android смартфоны не имеют GPIO портов и возможностей для прямого взаимодействия с электро-механическими устройствами\cite{AndroidSmarts}. Тем не менее, Android предоставляет два обходных решения. Первое заключается в поддержке взаимодействия с внешними устройствами по USB\cite{AndroidUSB}, что подходит не для всех устройств и внесет дополнительную сложность в реализацию прототипа. Вторым вариантом является взаимодействие в рамках подхода Android ADK\cite{AndroidUSB}, при этом в дополнении к устройству на базе Android для реализации прототипа замка необходимо использовать дополнительное решение, например, плату Arduino ADK. &

+/$-$\\

\hline
Скорость и удобство разработки &

Стандартным языком для разработки приложений для Android является язык Java, а стандартной IDE --- Eclipse\cite{Android}. Эти два фактора обеспечивают хорошие возможности для скоростной и комфортной разработки Android приложений. &

+\\

\hline
Вычислительные ресурсы &

Практически все современные Android смартфоны соответствуют необходимым физическим требованиям\cite{AndroidSmarts} в силу более высоких минимальных требования для работы ОС Android\cite{AndroidHR}. &

+\\

\hline
Беспроводное взаимодействие &

Практически все современные смартфоны на Android содержат WiFi и Bluetooth модули.\cite{AndroidSmarts}. &

+\\

\hline
Взаимодействие по локальной сети &

Практически все современные смартфоны на Android содержат WiFi модуль и не содержат порт Ethernet\cite{AndroidSmarts}.&

+\\

\hline
Стек TCP/IP &

ОС Android содержит полноценную реализацию стека TCP/IP, а Android SDK предоставляет все необходимые средства для создания TCP и UDP сокетов и взаимодействию по протоколам HTTP и HTTPS\cite{AndroidNetwork}. &

+\\

\hline
Многозадачность &

Android является многозадачной операционной системой, также предоставляя стандартные средства для разработки многопоточных приложений\cite{AndroidThreading}. &

+\\

\hline 
Документация разработчика и поддержка & 

Официальный сайт Android Developers содержит исчерпывающее количество документации и учебных материалов по разработке приложений на базе Android SDK. Также на тему разработки приложений для Android было сделано большое количество докладов на различных конференциях, написано огромное количество статей и обсуждений на различных форумах. &

+\\

\hline

Стоимость &

От \$160 \cite{AndroidBuy}&

\\

\hline
\caption{Android} % needs to go inside longtable environment
\end{longtable}
\egroup

\subsection{AllWinner A1X}
Семейство одноядерных однокристальных систем. На данный момент представлено решениями A10, A13, A10s и A31. Содержит одно ядро ARM Cortex-A8 как основной процессор CPU и графический процессор GPU Mali 400\cite{AllWinner}.

Самого по себе AllWinner A1X недостаточно для решения поставленной задачи, однако на его базе существует ряд интересных решений: BeagleBone\cite{BeagleBone}, Hackberry\cite{Hackberry}, Cubieboard\cite{Cubieboard}, Gooseberry\cite{Gooseberry} и ряд других. Для поставленной задачи наиболее подходящими вариантами являются решения, содержащие порты GPIO и доступные для заказа --- это CubieBoard и BeagleBone. Данные устройства весьма схожи между собой по техническим характеристикам, и были разработаны позже Raspberry Pi с целью составить достойную конкуренцию за счет более высоких аппаратно-вычислительных возможностей при сравнимой с Pi цене. Поэтому оценка AllWinner A1X будет даваться именно на основе возможностей CubieBoard и BeagleBone.

\bgroup
\def\arraystretch{1.5}%  1 is the default, change whatever you need

\begin{longtable}{| p{.2\textwidth} | p{.6\textwidth} | p{1cm} | p{1cm} |} 

\hline
Критерий & Комментарий & \rotatebox{90}{BeagleBone} & \rotatebox{90}{CubieBoard} \\
\hline
Взаимодействие с электро-механическими  устройствами &

BeagleBone и CubieBoard поддерживают прямое программирование портов GPIO. Для BeagleBone это возможно по умолчанию средствами языка Bash\cite{BeagleBoneGPIO}, для CubieBoard с начала придется поставить Linux в качестве ОС. &

+ &

+/$-$\\

\hline
Скорость и удобство разработки &

Предустановленной ОС для CubieBoard является Android, программирование которого возможно на Java в Eclipse IDE. Для BeagleBone предустановленной ОС является Angstrom\cite{Angstrom}, программирование которого возможно на JavaScrtipt в Cloud9 IDE\cite{BeagleBone}. При установке Linux на CubieBoard для написания программ можно использовать C/C++ и Eclipse IDE\cite{Cubieboard}. Все перечисленные варианты должны обеспечить достаточно высокую скорость и удобство разработки. &

+ &

+\\

\hline
Вычислительные ресурсы &

CubieBoard и BeagleBone обладают достаточными физическими характеристиками. BeagleBone: 720 МГц процессор, 256 МБ RAM, возможность подключения до 32 ГБ внешней памяти\cite{BeagleBone}. CubieBoard: 1 ГГц процессор, 1 ГБ RAM, 4 ГБ внутренней NAND памяти и возможность подключения до 32 ГБ внешней памяти\cite{Cubieboard}. &

+ &

+\\

\hline
Беспроводное взаимодействие &

Реализация взаимодействия по Bluetooth и WiFi теоретически возможна путем подключения к USB порту соответствующих адаптеров, однако на практике сообщается о проблемах и необходимости внесения патчей в ядро системы. &

+/$-$ &

+/$-$\\

\hline
Взаимодействие по локальной сети &

BeagleBone и CubieBoard имеют встроенный Ethernet порт и все необходимые средства для написания как клиентских, так и серверных приложений.&

+ &

+\\

\hline
Стек TCP/IP &

Angstrom, Android, Ubuntu как и все другие популярные варианты ОС для CubieBoard и BeagleBone имеют полноценную реализацию стека TCP/IP и предоставляют необходимые для прототипа сетевые возможности. &

+ &

+\\

\hline
Многозадачность &

Angstrom, Android, Ubuntu как и все другие популярные варианты ОС для CubieBoard и BeagleBone являются многозадачными операционными системами. &

+ &

+\\

\hline 
Документация разработчика и поддержка & 

CubieBoard и BeagleBone значительно проигрывают Raspberry Pi в качестве документации разработчика, количеству статей и обсуждений, размере и активности коммьюнити разработчиков, наличию сторонних расширений и приложений, а также другим смежным параметрам.

Например, на первой странице выдачи Google по запросу <<raspberry pi gpio>> можно найти ссылку на специально созданную для этого библиотеку raspberry-gpio-python\cite{RPiGPIOPython} и еще 8 обучающих статей по этому вопросу. Запрос <<beaglebone gpio>> --- в выдаче библиотека beaglebone-gpio для языка C и 4 статьи.
Запрос <<cubieboard gpio>> --- в выдаче только спецификации cubieboard и безуспешные обсуждения на форумах. &

+/$-$ &

$-$\\

\hline

Стоимость &

CubieBoard --- \$60\cite{CubieBoardBuy}. BeagleBone --- \$89\cite{BeagleBoneBuy}

&&

\\

\hline
\caption{AllWinner A1X} % needs to go inside longtable environment
\end{longtable}
\egroup

\section{Заключение}
Из сравнительного анализа и оценок для каждого решения следует:

\begin{enumerate}

\item Наиболее подходящим средством для реализации прототипа замка M-Block является Raspberry Pi, так как данное решение получило положительные оценки для всех критериев, не получив ни одной негативной оценки. 

\item Arduino не может рассматриваться как возможный вариант для решения, так как имеет несоответствия по критериям <<Вычислительные ресурсы>> и <<Стек TCP/IP>>. 

\item Android и AllWinner A1X стоит признать подходящими и весьма хорошими вариантами для прототипа, но проигрывающими в сравнении Raspberry Pi. Android --- из-за не возможности прямого программирования GPIO портов, а решения на базе AllWinner A1X --- из-за недостаточного качества документации и низкой активности коммьюнити.

\end{enumerate}

%%%%%%%%%%%%%%     Литература    %%%%%%%%%%%%%%%%%%%%%%%
%%%%%%%%%%%%%% ГОСТ Р 7.0.5-2008 %%%%%%%%%%%%%%%%%%%%%%%

% можно воспользоваться системой Bibtex и стилевым файлом gost704.bst,
% входящим в состав 
\begin{thebibliography}{99}
% электронный ресурс оформляется по общим правилам + URL и дата обращения
\bibitem{Arduino}
Официальный сайт Arduino [Электронный ресурс].
URL: \url{http://www.arduino.cc/} (дата обращения: 31.03.2013).

\bibitem{ArduinoGPIO}
Arduino Analog Input Pins // Официальный сайт Arduino [Электронный ресурс]. 
URL: \url{http://arduino.cc/en/Tutorial/AnalogInputPins} (дата обращения: 31.03.2013).

\bibitem{ArduinoMemory}
Arduino Memory // Официальный сайт Arduino [Электронный ресурс]. 
URL: \url{http://arduino.cc/en/Tutorial/Memory} (дата обращения: 31.03.2013).

\bibitem{ArduinoBT}
Arduino Bluetooth Shield // Официальный сайт Arduino [Электронный ресурс]. 
URL: \url{http://arduino.cc/en/Main/ArduinoBoardBluetooth} (дата обращения: 31.03.2013).

\bibitem{XBee}
XBee Shield Wiki // Seeed Wiki [Электронный ресурс]. 
URL: \url{http://www.seeedstudio.com/wiki/XBee\%C2\%AE_Shield} (дата обращения: 31.03.2013).

\bibitem{ArduinoWiFi}
Arduino WiFi Shield // Официальный сайт Arduino [Электронный ресурс].  
URL: \url{http://arduino.cc/en/Main/ArduinoWiFiShield} (дата обращения: 31.03.2013).

\bibitem{ArduinoEthernet}
Arduino Ethernet Shield // Официальный сайт Arduino.  
URL: \url{http://arduino.cc/en/Main/ArduinoEthernetShield} (дата обращения: 31.03.2013).

\bibitem{ArduinoServer}
WebServer // Официальный сайт Arduino [Электронный ресурс].  
URL: \url{http://arduino.cc/en/Tutorial/WebClient} (дата обращения: 31.03.2013).

\bibitem{ArduinoClient}
WiFi Twitter Client // Официальный сайт Arduino [Электронный ресурс].  
URL: \url{http://arduino.cc/en/Tutorial/WiFiTwitterClient} (дата обращения: 31.03.2013).

\bibitem{RTOSArduino}
RTOS for Arduino // Out There [Электронный ресурс].  
URL: \url{http://arduino.cc/en/Tutorial/WiFiTwitterClient} (дата обращения: 31.03.2013).

\bibitem{ArduinoBuy}
Arduino Starter Kit // Arduino Store [Электронный ресурс].  
URL: \url{http://store.arduino.cc/ww/index.php?main_page=product_info&cPath=2_23&products_id=185} (дата обращения: 31.03.2013).

%------------------------------

\bibitem{RaspberryPi}
Официальный сайт Raspberry Pi [Электронный ресурс].
URL: \url{http://www.raspberrypi.org/} (дата обращения: 31.03.2013).

\bibitem{RPIHardware}
RPi Hardware // eLinux.org [Электронный ресурс]. 
URL: \url{http://elinux.org/RPi_Hardware} (дата обращения: 31.03.2013).

\bibitem{RPiGPIOPython}
Python library for GPIO access on Raspberry Pi // Google Code [Электронный ресурс].
URL: \url{https://code.google.com/p/raspberry-gpio-python/} (дата обращения: 31.03.2013).

\bibitem{RPiFAQ}
FAQs // Официальный сайт Raspberry Pi [Электронный ресурс].
URL: \url{http://www.raspberrypi.org/faqs} (дата обращения: 31.03.2013).

\bibitem{Raspbian}
Официальный сайт Raspbian [Электронный ресурс].
URL: \url{http://www.raspbian.org/} (дата обращения: 31.03.2013).

\bibitem{RPiBuy}
Raspberry Pi // RS Online [Электронный ресурс].
URL: \url{http://uk.rs-online.com/web/generalDisplay.html?id=raspberrypi} (дата обращения: 31.03.2013).

%------------------------------------

\bibitem{Android}
Официальный сайт Android Developers [Электронный ресурс].
URL: \url{http://developers.android.com} (дата обращения: 31.03.2013).

\bibitem{AndroidSmarts}
Сравнение Android устройств // Википедия [Электронный ресурс].
URL: \url{http://ru.wikipedia.org/wiki/Сравнение_устройств_с_Android} (дата обращения: 31.03.2013).

\bibitem{AndroidUSB}
USB Host and Accessory // Android Developers [Электронный ресурс].
URL: \url{http://developer.android.com/guide/topics/connectivity/usb/index.html} (дата обращения: 31.03.2013).

\bibitem{AndroidHR}
Android Hardware Requirments // Android Hardwares [Электронный ресурс].
URL: \url{http://androidhardwares.com/android-hardware-development/android-hardware-requirements/} (дата обращения: 31.03.2013).

\bibitem{AndroidNetwork}
Performing Network Operations // Android Developers [Электронный ресурс].
URL: \url{http://developer.android.com/training/basics/network-ops/index.html} (дата обращения: 31.03.2013).

\bibitem{AndroidNetwork}
Performing Network Operations // Android Developers [Электронный ресурс].
URL: \url{http://developer.android.com/training/basics/network-ops/index.html} (дата обращения: 31.03.2013).

\bibitem{AndroidThreading}
Painless Threading // Android Developers [Электронный ресурс].
URL: \url{http://android-developers.blogspot.com/2009/05/painless-threading.html} (дата обращения: 31.03.2013).

\bibitem{AndroidBuy}
Самый дешевый в мире Android смартфон // АйМобилка [Электронный ресурс].
URL: \url{http://icellphone.ru/kitajcy-pokazali-samyj-deshevyj-v-mire-android-smartfon.html} (дата обращения: 31.03.2013).

%--------------------------------------------------

\bibitem{AllWinner}
AllWinner A Serial // Официальный сайт AllWinner Technology [Электронный ресурс].
URL: \url{http://www.allwinnertech.com/en/product/A-Serial.html} (дата обращения: 31.03.2013).

\bibitem{Cubieboard}
Официальный сайт Cubieboard [Электронный ресурс].
URL: \url{http://linux-sunxi.org/Cubieboard} (дата обращения: 31.03.2013).

\bibitem{BeagleBone}
Официальный сайт GooseBerry [Электронный ресурс].
URL: \url{http://beagleboard.org/bone} (дата обращения: 31.03.2013).

\bibitem{Gooseberry}
Официальный сайт Gooseberry [Электронный ресурс].
URL: \url{http://gooseberry.atspace.co.uk/} (дата обращения: 31.03.2013).

\bibitem{Hackberry}
Официальный сайт Hackberry [Электронный ресурс].
URL: \url{https://www.miniand.com/products/Hackberry\%20A10\%20Developer\%20Board} (дата обращения: 31.03.2013).

\bibitem{BeagleBoneGPIO}
Accessing GPIO // BeagleBone for Dummies [Электронный ресурс].
URL: \url{http://bbfordummies.blogspot.ru/2009/07/1.html} (дата обращения: 31.03.2013).

\bibitem{Angstrom}
Angstrom Distribution // BeagleBoard.org [Электронный ресурс].
URL: \url{http://beagleboard.org/project/angstrom/} (дата обращения: 31.03.2013).

\bibitem{BeagleBoneBuy}
BeagleBone A6 // Adafruit [Электронный ресурс].
URL: \url{https://www.adafruit.com/products/513} (дата обращения: 31.03.2013).

\bibitem{CubieBoardBuy}
CubieBoard // DFRobot [Электронный ресурс].
URL: \url{http://www.dfrobot.com/index.php?route=product/product&filter_name=cubieboard&product_id=881#.UVxjTZMqwUR} (дата обращения: 31.03.2013).

\end{thebibliography}

\endarticle  %  Команда завершения статьи

\end{document}