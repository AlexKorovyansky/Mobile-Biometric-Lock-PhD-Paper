\chapter{Программная реалиазация} \label{chapt3}

\section{Введение} \label{sect3_1}

Для тестирования и апробации разработанной модели был реализован прототип СКУД, строго соответствующий модели. Перед разработкой прототипа был проведен подробный обзор существующих решений с целью выбора наиболее подходящих средств для реализации прототипа.

\section{Обзор решений для реализации прототипа контроллера замка} \label{sect3_2}

Для сравнительного анализа в условиях данной работы были выдвинуты следующие критерии:
\begin{enumerate}
  \item \textbf{Взаимодействие с электро-механическим устройствами.}\hfill ~\linebreak
        Контроллер замка должен взаимодействовать с электро-механическими устройствами. Поэтому средство для реализации прототипа контроллера замка должно обязательно предоставлять программно-аппаратные возможности для взаимодействия с электро-механическими устройствами.

  \item \textbf{Беспроводное взаимодействие.}\hfill ~\linebreak
        Контроллер замка должен устанавливать беспроводное соединение с мобильным приложением, поэтому устройство должно поддерживать как минимум один современных способов беспроводной передачи данных на небольших расстояний.

  \item \textbf{Многозадачность.}\hfill ~\linebreak
        Контроллер замка замка должен одновременно коммуницировать возможно с несколькими мобильными приложениями. Поэтому средство для реализации прототипа контроллера замка должен предоставлять средства для реализации многозадачности.

  \item \textbf{Скоростная и качественная разработка.}\hfill ~\linebreak
        Скорость разработки без потери ее качества также является крайне значимым критерием. Поэтому средство для реализации прототипа контроллера замка должно поддерживать программирование на высокоуровневом языке программировании, с использованием удобной и эффективной IDE и других инструментов разработки. Так важным параметром является качество документации, количество статей и обсуждений в Интернет и поддержка со стороны производителя.
\end{enumerate} 

Были рассмотрены существующие решения: Arduino, Raspberry Pi, устройства на базе Android, CubieBoard и BeagleBone.

\bgroup
\def\arraystretch{1.5}%  1 is the default, change whatever you need
  \begin{longtable}{| p{.2\textwidth} | p{.6\textwidth} | c |} 
  \caption{Arduino} % needs to go inside longtable environment
    \hline
      Критерий & Комментарий & Оценка \\
    \hline
      Взаимодействие с электро-механическими  устройствами

      & Arduino поддерживает прямое программирование портов GPIO, является одной из его самых сильных сторон. 
      
      & + \\
    \hline
      Беспроводное взаимодействие

      & Arduino позволяет реализовать беспроводное взаимодействие по стандартам Bluetooth, WiFi и NFC за счет использования специальных плат или подключения дополнительных соответствующих шилдов. Так например взаимодействие по Bluetooth возможно за счет использование платы Arduino BT, или подключения шилда XBee.
      
      & + \\
    \hline
      Многозадачность

      & Arduino не имеет встроенных средств для обеспечения многозадачности. Поддержка многозадачности возможна только за счет использования специальных сторонних библиотек или RTOS. Однако, оба варианта несут в себе значительно увлечение сложности реализации. Поэтому, в целом, многозадачность в Arduino является весьма проблемным вопросом с большим количеством <<подводных камней>>.

      & $-$ \\
    \hline
      Скоростная и качественная разработка
      
      & Язык Processing, а также IDE для программирования Arduino являются достаточно хорошими и проверенными решениями, но значительно проигрывают в скорости и удобстве разработки современными высокоуровневыми языкам программирования, и соответствующим им IDE. 

      Офциальный сайт Arduino предоставляет большое количество обучающих материалов и действительно хорошую документацию разработчика. Более того вокруг платформы Arduino существует большое коммьюнити активных разработчиков, что также предоставляет хорошую возможность для решения вопросов, возникающих в процессе разработки.

      & +/$-$ \\
    \hline
  \end{longtable}
\egroup

\begin{table} [htbp]
  \centering
  \parbox{15cm}{\caption{Название таблицы}\label{Ts0Sib}}
%  \begin{center}
  \begin{tabular}{| p{3cm} || p{3cm} | p{3cm} | p{4cm}l |}
  \hline
  \hline
  Месяц   & \centering $T_{min}$, К & \centering $T_{max}$, К &\centering  $(T_{max} - T_{min})$, К & \\
  \hline
  Декабрь &\centering  253.575   &\centering  257.778    &\centering      4.203  &   \\
  Январь  &\centering  262.431   &\centering  263.214    &\centering      0.783  &   \\
  Февраль &\centering  261.184   &\centering  260.381    &\centering     $-$0.803  &   \\
  \hline
  \hline
  \end{tabular}
%  \end{center}
\end{table}

%\newpage
%============================================================================================================================

\section{Параграф - два} \label{sect3_2}

Некоторый текст.

%\newpage
%============================================================================================================================

\section{Параграф с подпараграфами} \label{sect3_3}

\subsection{Подпараграф - один} \label{subsect3_3_1}

Некоторый текст.

\subsection{Подпараграф - два} \label{subsect3_3_2}

Некоторый текст.

\clearpage