\chapter{Программная реалиазация} \label{chapt3}

\section{Введение} \label{sect3_1}

Для тестирования и апробации разработанной модели был реализован прототип СКУД, строго соответствующий модели. Перед разработкой прототипа был проведен подробный обзор существующих решений с целью выбора наиболее подходящих средств для реализации прототипа.

\section{Обзор решений для реализации прототипа контроллера замка} \label{sect3_2}

Для сравнительного анализа в условиях данной работы были выдвинуты следующие критерии:
\begin{enumerate}
  \item \textbf{Взаимодействие с электро-механическим устройствами.}\hfill ~\linebreak
        Контроллер замка должен взаимодействовать с электро-механическими устройствами. Поэтому средство для реализации прототипа контроллера замка должно обязательно предоставлять программно-аппаратные возможности для взаимодействия с электро-механическими устройствами.

  \item \textbf{Беспроводное взаимодействие.}\hfill ~\linebreak
        Контроллер замка должен устанавливать беспроводное соединение с мобильным приложением, поэтому устройство должно поддерживать как минимум один современных способов беспроводной передачи данных на небольших расстояний.

  \item \textbf{Многозадачность.}\hfill ~\linebreak
        Контроллер замка замка должен одновременно коммуницировать возможно с несколькими мобильными приложениями. Поэтому средство для реализации прототипа контроллера замка должен предоставлять средства для реализации многозадачности.

  \item \textbf{Скоростная и качественная разработка.}\hfill ~\linebreak
        Скорость разработки без потери ее качества также является крайне значимым критерием. Поэтому средство для реализации прототипа контроллера замка должно поддерживать программирование на высокоуровневом языке программировании, с использованием удобной и эффективной IDE и других инструментов разработки. Так важным параметром является качество документации, количество статей и обсуждений в Интернет и поддержка со стороны производителя.
\end{enumerate} 

Были рассмотрены существующие решения: Arduino, Raspberry Pi, устройства на базе Android, CubieBoard и BeagleBone.

\bgroup %Arduino
\def\arraystretch{1.5}%  1 is the default, change whatever you need
  \begin{longtable}{| p{.2\textwidth} | p{.6\textwidth} | c |} 
  \caption{Arduino} % needs to go inside longtable environment
  \hline
    Критерий & Комментарий & Оценка \\
  \hline
    Взаимодействие с электро-механическими устройствами

    & Arduino поддерживает прямое программирование портов GPIO, является одной из его самых сильных сторон. 
    
    & + \\
  \hline
    Беспроводное взаимодействие

    & Arduino позволяет реализовать беспроводное взаимодействие по стандартам Bluetooth, WiFi и NFC за счет использования специальных плат или подключения дополнительных соответствующих шилдов. Так например взаимодействие по Bluetooth возможно за счет использование платы Arduino BT, или подключения шилда XBee.
    
    & + \\
  \hline
    Многозадачность

    & Arduino не имеет встроенных средств для обеспечения многозадачности. Поддержка многозадачности возможна только за счет использования специальных сторонних библиотек или RTOS. Однако, оба варианта несут в себе значительно увлечение сложности реализации. Поэтому, в целом, многозадачность в Arduino является весьма проблемным вопросом с большим количеством <<подводных камней>>.

    & $-$ \\
  \hline
    Скоростная и качественная разработка
    
    & Язык Processing, а также IDE для программирования Arduino являются достаточно хорошими и проверенными решениями, но значительно проигрывают в скорости и удобстве разработки современными высокоуровневыми языкам программирования, и соответствующим им IDE. 

    Офциальный сайт Arduino предоставляет большое количество обучающих материалов и действительно хорошую документацию разработчика. Более того вокруг платформы Arduino существует большое коммьюнити активных разработчиков, что также предоставляет хорошую возможность для решения вопросов, возникающих в процессе разработки.

    & +/$-$ \\
  \hline
  \end{longtable}
\egroup %Arduino

\pagebreak

\bgroup %Raspberry Pi
\def\arraystretch{1.5}%  1 is the default, change whatever you need
  \begin{longtable}{| p{.2\textwidth} | p{.6\textwidth} | c |} 
  \caption{Raspberry Pi} % needs to go inside longtable environment
  \hline
    Критерий & Комментарий & Оценка \\
  \hline
    Взаимодействие с электро-механическими устройствами

    & Raspberry Pi поддерживает прямое программирование портов GPIO. Также стоит отметить, факт существования проекта raspberry-gpio-python позволяющего управлять GPIO на высокоуровневом языке Python.  
    & + \\
  \hline
    Беспроводное взаимодействие

    & Реализация взаимодействия по Bluetooth и WiFi возможна путем подключения к USB порту Raspberry Pi WiFi/Bluetooth адаптера. Также заявляется о успешной поддержке NFC.
    
    & + \\
  \hline
    Многозадачность

    & Raspbian, также как и Debian, является многозадачной операционной системой и более того предоставляет средства для разработки многопоточных приложений.

    & + \\
  \hline
    Скоростная и качественная разработка
    
    & Официально поддерживаемым языком для разработки приложений на Raspberry Pi является Python, для разработки на котором большое количество различных IDE.

    Для Raspberry Pi в сети Интернет можно найти большое количество документации и различных учебных материалов. Также вокруг Raspberry Pi сформировалось очень большое и очень активное сообщество разработчиков.

    & + \\
  \hline
  \end{longtable}
\egroup %Raspberry Pi

\pagebreak

\bgroup %Android
\def\arraystretch{1.5}%  1 is the default, change whatever you need
  \begin{longtable}{| p{.2\textwidth} | p{.6\textwidth} | c |} 
  \caption{Устройства на базе Android} % needs to go inside longtable environment
  \hline
    Критерий & Комментарий & Оценка \\
  \hline
    Взаимодействие с электро-механическими устройствами

    & Существующие Android смартфоны не имеют GPIO портов и возможности прямого взаимодействия с электро-механическими устройствами. Тем не менее, Android предоставляет два обходных решения — Usb Accessory and Android ADK.

    & +/$-$ \\
  \hline
    Беспроводное взаимодействие

    & Практически все современные смартфоны на Android содержат WiFi и Bluetooth модули. Некоторые модели, например Samsung Galaxy Nexus, имеют также и NFC.
    
    & + \\
  \hline
    Многозадачность

    & Android является многозадачной операционной системой, более того Android SDK предоставляет простые и удобные средства для разработки многопоточных приложений.

    & + \\
  \hline
    Скоростная и качественная разработка
    
    & Стандартным языком для разработки приложений для Android является язык Java, а стандартной IDE — Eclipse.

    Официальный сайт Android Developers предоставляет исчерпывающее количество документации и учебных материалов по разработке приложений на базе Android SDK. 

    Также на тему разработки приложений для Android было сделано большое количество докладов на различных конференциях, написано огромное количество статей и обсуждений на различных форумах.

    & + \\
  \hline
  \end{longtable}
\egroup %Android

\pagebreak

\bgroup %BeagleBone and CubieBoard
\def\arraystretch{1.5}%  1 is the default, change whatever you need
  \begin{longtable}{| p{.2\textwidth} | p{.6\textwidth} | p{1cm} | p{1cm} |} 
  \caption{BeagleBone и CubieBoard} % needs to go inside longtable environment
  \hline
    Критерий & Комментарий & \rotatebox{90}{BeagleBone} & \rotatebox{90}{CubieBoard} \\
  \hline
    Взаимодействие с электро-механическими устройствами

    & BeagleBone и CubieBoard поддерживают прямое программирование портов GPIO. Для BeagleBoard это возможно по-умолчанию средствами языка Bash, для CubieBoard с начала придется поставить Linux в качестве ОС.

    & + & +/$-$ \\
  \hline
    Беспроводное взаимодействие

    & Реализация взаимодействия по Bluetooth и WiFi теоретически возможна путем подключения к USB порту соответствующих адаптеров, однако на практике сообщается о проблемах и необходимости внесения патчей в ядро системы.
    
    & + & +/$-$ \\
  \hline
    Многозадачность

    & Armstrong, Android, Ubuntu как и все другие популярные варианты ОС для CubieBoard и BeagleBone являются многозадачными операционными системами.

    & + & + \\
  \hline
    Скоростная и качественная разработка
    
    & CubieBoard и BeagleBone проигрывают Raspberry Pi в качестве документации разработчика, числе обучающих материалов, размере и активности коммьюнити разработчиков, наличию сторонних расширений и приложений, а также другим смежным параметрам. 

    & +/$-$ & +/$-$\\
  \hline
  \end{longtable}
\egroup %BeagleBone and CubieBoard

По результатам сравнительного анализа возможных решений в качестве средства для реализации прототипа контроллера замка был выбран Raspberry Pi.

\section{Обзор решений для реализации прототипа мобильного приложения} \label{sect3_3}

Для сравнительного анализа в условиях данной работы были выдвинуты следующие критерии:

\begin{enumerate}
  \item \textbf{Средства HTTP.}\hfill ~\linebreak
        Взаимодействие мобильного приложения с веб-приложение должно быть реализовано в рамках подхода REST. Важным критерием при выборе мобильной платформы является наличие библиотек или инструментов для быстрой и эффективной реализации REST клиента.

  \item \textbf{Средства безопасности.}\hfill ~\linebreak
        Протокол системы подразумевает собой генерацию ключей в приложении на мобильном устройстве. Для реализации этой функции необходимо присутствие библиотек шифрования и поддержка различных криптографических алгоритмов.

  \item \textbf{Хранение данных.}\hfill ~\linebreak
        Мобильное приложение должно хранить токены всех замков полученные от сервера в памяти мобильного устройства. Для реализации этого функционала большим плюсом будет программная и аппаратная поддержка внутреннего и внешнего хранилища данных и наличие библиотек для работы с базой данных.

  \item \textbf{Поддержка беспроводных технологий.}\hfill ~\linebreak
        Мобильное приложение должно взаимодействовать с контроллером замка по технологиям Bluetooth или Wiki. Дополнительным преимуществом является поддержка технологии NFC.

  \item \textbf{Скоростная и качественная разработка.}\hfill ~\linebreak
        Скорость разработки без потери ее качества также является крайне значимым критерием. Поэтому средство для реализации прототипа мобильного приложения должно поддерживать программирование на высокоуровневом языке программировании, с использованием удобной и эффективной IDE и других инструментов разработки. Также важным параметром является качество документации, количество статей и обсуждений в Интернет и поддержка со стороны производителя.
\end{enumerate} 

Были рассмотрены следующие решения: iOS, Android и Symbian.

\bgroup %iOS
\def\arraystretch{1.5}%  1 is the default, change whatever you need
  \begin{longtable}{| p{.2\textwidth} | p{.6\textwidth} | c |} 
  \caption{iOS} % needs to go inside longtable environment
  \hline
    Критерий & Комментарий & Оценка \\
  \hline
    Средства HTTP

    & В Интернете можно найти статьи с примерами создания Rest API Client и большое количество библиотек для взаимодействия с сервером: RestKit, AFNetworking и др. 
    & + \\
  \hline
    Средства безопасности

    & Для шифрования и генерации ключей в iOS используется библиотека CommonCrypto
    
    & + \\
  \hline
    Хранение данных

    & IOS имеет поддержку SQLite, хранения настроек приложения и предоставляет фреймоворк для хранения данных Core Data

    & + \\
  \hline
    Поддержка беспроводных технологий 
    
    & Начиная с самой первой модели iPhone была включена поддержка WiFi и Bluetooth.  Однако даже самые последние модели iPhone не имеют поддержки технологии NFC

    & +/$-$ \\
  \hline
    Скоростная и качественная разработка
    
    & Инстурменты разработки доступны только для операционной системы OS X. Для тестирования на реальном устройстве необходимо приобрести лицензию разработчика. На официальном сайте Apple Developers доступны обширная документация и обучающие статьи для начинающих разработчиков.
    & +/$-$ \\
  \hline

  \end{longtable}
\egroup %iOS

\pagebreak

\bgroup %Android
\def\arraystretch{1.5}%  1 is the default, change whatever you need
  \begin{longtable}{| p{.2\textwidth} | p{.6\textwidth} | c |} 
  \caption{Android} % needs to go inside longtable environment
  \hline
    Критерий & Комментарий & Оценка \\
  \hline
    Средства HTTP

    & Android SDK предоставляет все необходимые средства для взаимодействия по протоколам HTTP и HTTPS в рамках стандартного подхода разработки Android приложений на языке Java. Также существуют библиотеки для качественной и быстрой реализации Rest Client, например Retrofit

    & + \\
  \hline
    Средства безопасности

    & Использование стандартных библиотек java.security и javax.crypto
    
    & + \\
  \hline
    Хранение данных

    & Доступны различные методы хранения информации Shared Preferences, внутренняя и внешняя память телефона и SQLite

    & + \\
  \hline
    Поддержка беспроводных технологий 
    
    & Практически все современные смартфоны на Android содержат WiFi и Bluetooth модули. Некоторые модели, например Samsung Galaxy Nexus, имеют также и NFC.

    & + \\
  \hline
    Скоростная и качественная разработка
    
    & Разработку под Android можно вести в различных ОС, с использованием языка Java. Android SDK и возможности большого количества сторонних библиотек позволяют вести скоростную и качественную разработку прототипа мобильного приложения.

    & + \\
  \hline

  \end{longtable}
\egroup %Android

\pagebreak

\bgroup %Symbian
\def\arraystretch{1.5}%  1 is the default, change whatever you need
  \begin{longtable}{| p{.2\textwidth} | p{.6\textwidth} | c |} 
  \caption{Symbian} % needs to go inside longtable environment
  \hline
    Критерий & Комментарий & Оценка \\
  \hline
    Средства HTTP

    & Symbian имеет встроенные средства для работы по протоколу HTTP, но не имеет решений аналогичных Retrofit.
    & +/$-$\\
  \hline
    Средства безопасности

    & Возможно использование библиотеки OpenSSL.
    
    & + \\
  \hline
    Хранение данных

    & Существуют библиотеки для работы с базой данных 

    & + \\
  \hline
    Поддержка беспроводных технологий 
    
    & Наличие WiFi и Bluetooth модулей. Некоторые устройства поддерживают NFC : Nokia 700, Nokia C7-00, Nokia Oro

    & + \\
  \hline
    Скоростная и качественная разработка
    
    & Основной язык разработки C++. Официальная IDE Carbide. Есть поддержка эмулятора. Популярным ресурсом является сайт Nokia Developers, при этом платформа официально больше не поддерживается.

    & $-$ \\
  \hline

  \end{longtable}
\egroup %Symbian

По результатам сравнительного анализа возможных решений в качестве средства для реализации прототипа контроллера замка был выбран Android.

\clearpage