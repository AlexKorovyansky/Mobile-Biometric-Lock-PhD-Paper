\chapter{Программная реалиазация} \label{chapt3}

\section{Введение} \label{sect3_1}

Для тестирования и апробации разработанной модели был реализован прототип СКУД, строго соответствующий модели. Перед разработкой прототипа был проведен подробный обзор существующих решений с целью выбора наиболее подходящих средств для реализации прототипа.

\section{Обзор решений для реализации прототипа контроллера замка} \label{sect3_2}

Для сравнительного анализа в условиях данной работы были выдвинуты следующие критерии:
\begin{enumerate}
  \item \textbf{Взаимодействие с электро-механическим устройствами.}\hfill ~\linebreak
        Контроллер замка должен взаимодействовать с электро-механическими устройствами. Поэтому средство для реализации прототипа контроллера замка должно обязательно предоставлять программно-аппаратные возможности для взаимодействия с электро-механическими устройствами.

  \item \textbf{Беспроводное взаимодействие.}\hfill ~\linebreak
        Контроллер замка должен устанавливать беспроводное соединение с мобильным приложением, поэтому устройство должно поддерживать как минимум один современных способов беспроводной передачи данных на небольших расстояний.

  \item \textbf{Многозадачность.}\hfill ~\linebreak
        Контроллер замка замка должен одновременно коммуницировать возможно с несколькими мобильными приложениями. Поэтому средство для реализации прототипа контроллера замка должен предоставлять средства для реализации многозадачности.

  \item \textbf{Скоростная и качественная разработка.}\hfill ~\linebreak
        Скорость разработки без потери ее качества также является крайне значимым критерием. Поэтому средство для реализации прототипа контроллера замка должно поддерживать программирование на высокоуровневом языке программировании, с использованием удобной и эффективной IDE и других инструментов разработки. Так важным параметром является качество документации, количество статей и обсуждений в Интернет и поддержка со стороны производителя.
\end{enumerate} 

\bgroup
\def\arraystretch{1.5}%  1 is the default, change whatever you need

\begin{longtable}{| p{.2\textwidth} | p{.6\textwidth} | c |} 

\hline
Критерий & Комментарий & Оценка \\
\hline
Взаимодействие с электро-механическими  устройствами &

Arduino поддерживает прямое программирование портов GPIO, что является одной из его самых сильных сторон\cite{ArduinoGPIO}. &

+ \\

\hline
Скорость и удобство разработки &

Язык Processing, а также IDE для программирования Arduino являются достаточно хорошими и проверенными решениями, но значительно проигрывают в скорости и удобстве разработки современным высокоуровневыми языкам программирования, и соответствующим им IDE.

Также при использовании большого количества дополнительных шилдов может возникнуть серьезная проблема нехватки GPIO разъемов на плате и необходимости взаимоисключения шилдов. &

+/$-$\\

\hline
Вычислительные ресурсы &

Решения на базе Arduino имеют очень ограниченную и явно недостаточную для прототипа вычислительную мощность. Частота микроконтроллера не превышает 16 МГц, а размер оперативой памяти - 8 КБ\cite{Arduino Memory}. &

$-$\\

\hline
Беспроводное взаимодействие &

Arduino позволяет реализовать беспроводное взаимодействие по стандартам Bluetooth и WiFi за счет использования специальных плат или подключения дополнительных соответствующих шилдов. Так, например, взаимодействие по Bluetooth возможно за счет использование платы Arduino BT{ArduinoBT} или подключения шилда XBee\cite{XBee}. &

+\\

\hline
Взаимодействие по локальной сети &

Путем подключения дополнительных шилдов Ethernet\cite{ArduinoEthernet} и WiFi\cite{ArduinoWiFi} возможна реализация взаимодействия по соответствующему стандарту локальной сети. При этом плата Arduino может выступать как в роли клиента\cite{ArduinoClient}, так и сервера\cite{ArduinoServer}.&

+\\

\hline
Стек TCP/IP &

Стандартные библиотеки Arduino характеризуются очень низкоуровневым подходом к реализации взаимодействия по протоколам стека TCP/IP и отсутствием поддержки некоторых протоколов или определенных их возможностей. Например, Arduino не поддерживает взаимодействие по протоколу HTTPS по причине недостаточной вычислительной мощности. &

$-$\\

\hline
Многозадачность &

Arduino не имеет встроенных средств для обеспечения многозадачности. Поддержка многозадачности возможна только за счет использования специальных сторонних библиотек или RTOS\cite{RTOSArduino}. Однако, оба варианта приведут к значительному усложнению реализации. В итоге, многозадачность в Arduino является весьма проблемным вопросом с большим количеством <<подводных камней>>. &

$-$\\

\hline 
Документация разработчика и поддержка & 

Официальный сайт Arduino предоставляет большое количество обучающих материалов и хорошую документацию разработчика. Также вокруг платформы существует большое коммьюнити активных разработчиков, что значительно облегчает процесс решения сложных вопросов, возникающих в процессе разработки. &

+\\

\hline

Стоимость &

\EUR{80} \cite{ArduinoBuy} &

\\

\hline
\caption{Arduino} % needs to go inside longtable environment
\end{longtable}
\egroup

\begin{table} [htbp]
  \centering
  \parbox{15cm}{\caption{Название таблицы}\label{Ts0Sib}}
%  \begin{center}
  \begin{tabular}{| p{3cm} || p{3cm} | p{3cm} | p{4cm}l |}
  \hline
  \hline
  Месяц   & \centering $T_{min}$, К & \centering $T_{max}$, К &\centering  $(T_{max} - T_{min})$, К & \\
  \hline
  Декабрь &\centering  253.575   &\centering  257.778    &\centering      4.203  &   \\
  Январь  &\centering  262.431   &\centering  263.214    &\centering      0.783  &   \\
  Февраль &\centering  261.184   &\centering  260.381    &\centering     $-$0.803  &   \\
  \hline
  \hline
  \end{tabular}
%  \end{center}
\end{table}

%\newpage
%============================================================================================================================

\section{Параграф - два} \label{sect3_2}

Некоторый текст.

%\newpage
%============================================================================================================================

\section{Параграф с подпараграфами} \label{sect3_3}

\subsection{Подпараграф - один} \label{subsect3_3_1}

Некоторый текст.

\subsection{Подпараграф - два} \label{subsect3_3_2}

Некоторый текст.

\clearpage