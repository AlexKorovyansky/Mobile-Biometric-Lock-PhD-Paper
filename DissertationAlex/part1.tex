\chapter{Системы контроля и управления доступом} \label{chapt1}

\section{Введение} \label{sect1_1}

В широком смысле, \textbf{система контроля и управления доступом (СКУД)} — это совокупность программно-аппаратных технических средств безопасности, имеющих целью ограничение и регистрацию доступа субъектов (людей, транспорта и др.) к определенному контролируемому множеству объектов (двери, ворота, ячейки камер хранения, и др.). 

Наиболее распространенным на практике вариантом пары субъект-объект является пара <<человек-дверь>>, и даже простейший дверной замок и обычный ключ можно отнести к данному типу СКУД. При этом существует большое количество других более технологически-сложных и промышленно значимых СКУД.

Любая СКУД предполагает использование определенного \textbf{идентификатора}, при этом дл различных СКУД в роли идентификатора могут использоваться объекты совершенно разного типа. Например обычный ключ, пароль от кодового замка, смарт-карта, отпечаток пальца — все эти объекты являются идентификаторами для различных существующих на данный момент СКУД.


В ГОСТ Р 51241-2008 подробно описаны классификация и принципы работы промышленных СКУД. Согласно ГОСТ СКУД классифицируют по:
\begin{itemize}
  \item Cпособу управления.
  \item Числу контролируемых точек доступа.
  \item Функциональным характеристикам.
  \item Уровню защищенности системы от несанкционированного доступа к информации.
\end{itemize}

По способу управления СКУД подразделяют на:
\begin{itemize}
  \item Автономные — для управления одним или несколькими УПУ без передачи информации на центральное устройство управления и контроля со стороны оператора.
  \item Централизованные (сетевые) — для управления УПУ с обменом информацией с центральным пультом и контролем и управлением системой со стороны центрального устройства управления.
  \item Универсальные (сетевые) — включающие в себя функции как автономных, так и сетевых систем, работающие в сетевом режиме под управлением центрального устройства управления и переходящие в автономный режим при возникновении отказов в сетевом оборудовании, центральном устройстве или обрыве связи.
\end{itemize}

По числу контролируемых точек доступа:
\begin{itemize}
  \item Малой емкости (не более 84 точек).
  \item Cредней емкости (от 84 до 256 точек).
  \item Большой емкости (более 256 точек).
\end{itemize}

По функциональным характеристикам СКУД подразделяют на три класса:
\begin{itemize}
  \item 1-й — системы с ограниченными функциями.
  \item 2-й — системы с расширенными функциями.
  \item 3-й — многофункциональные системы.
\end{itemize}

По способу запирания:
\begin{itemize}
  \item Электромеханические замки.
  \item Электромагнитные замки.
  \item Электромагнитные защелки.
  \item Механизмы привода ворот.
\end{itemize} 

По способу считывания идентификационных признаков:
\begin{itemize}
  \item С ручным вводом.
  \item Контактные.
  \item Бесконтактные.
  \item Биометрические.
  \item Комбинированные.
\end{itemize} 

По типу идентификаторов:
\begin{itemize}
  \item Механические.
  \item Магнитные.
  \item Оптические.
  \item Электронные контактные.
  \item Электронные радиочастотные.
  \item Акустические.
  \item Комбинированные.
\end{itemize} 

Стоит отметить, что ГОСТ рассматривает СКУД только пределах промыщленного сегменте. В тоже время для личного использования, в офисах компаний малого и среднего бизнеса, образовательных и медицинских учреждениях в подавляющем большинстве случаев используются системы основанные на обыкновенных ключах. Причины этого заключается в сложности и высокой стоимости современных моделей СКУД в уже существующую инфраструктуру зданий, а также в экономической невыгодности их использования данных моделей СКУД при строительстве новых объектов. 

Таким образом существующие на данный момент модели СКУД имеют недостаточно высокую эффективность в крайне значимых для народного-хозяйства областях. В тоже время безопасность и удобство использования СКУД основанных на обыкновенных ключах является крайне низкой, что имеет негативное влияние для народного-хозяйства, так как малый и средний бизнес, сфера образования и здравоохранения являются ключевыми для экономики государства.

\section{Целесообразность создания новой модели мобильной СКУД} \label{sect1_2}

После детального изучения систем контроля доступа как объекта исследования, существующих систем контроля доступа и требований со стороны ГОСТ автором данной работы было сделано предположение о возможности создания новой модели СКУД, имеющей большую эффективность для сферы персонального использования, малого и среднего бизнеса, образовательных и медицинских учреждений, а также возможно для других объектов имеющих высокое народно-хозяйственное значение.

Идея использования смартфона к качестве кредитной карты уже успешно работает на практике и активно поддерживается и внедряется в том числе нескольких крупных российских банках. Этот факт стал причиной исследования и поиска новой модели СКУД именно в области мобильных технологий и мобильных устройств.

\medskip

\textbf{Мобильная СКУД} — СКУД, базовым идентификатором в которой является мобильное устройство пользователя.

Очевидно, модель мобильной СКУД содержит в себе возможность использования электронных ключей. Современные мобильные устройства позволяют реализовать все необходимые криптографические алгоритмы зафиксированные в рамках требований к СКУД в ГОСТ. В силу данных фактов, автором работы была сформулирована гипотеза о возможности разработка принципиально новой модели мобильной СКУД, непротиворечащей требованиям ГОСТ, и имеющей более высокую эффективность для некоторых областей народного хозяйства.

Следуя идее биометрических систем, в которых в качестве идентификатора используются биометрические фактор (например, отпечаток пальца) — в качестве смежной возможности для мобильных СКУД была определена возможность проведения биометрической верификации. То есть в необходимых случаях устройство пользователя может выполнять роль идентификатора только после прохождения пользователем биометрической верификации.

\medskip

\textbf{Биометрическая мобильная СКУД} — СКУД, базовым идентификатором в которой является мобильное устройство пользователя, при этом администратор системы имеет возможность требования биометрической верификации пользователя для определенных объектов СКУД. Другими словами для определенного списка объектов, мобильное устройство может выполнять роль идентификатора, только после успешного прохождения пользователем биометрической верификации.

Современные мобильные устройства позволяют производить захват изображения с камеры и аудио с микрофона, получать показания акселерометра, гирокомпаса и многих других сенсоров в режиме реального времени. В силу данных обстоятельств, мобильное устройство является уже готовым и практически идеальным устройством для захвата биометрических данных пользователя, что позволяет отказаться от использования дорогостоящих и сложных во внедрении аппаратных решений для захвата биометрических данных.

Таким образом, перед автором данной работы была поставлена задача разработки новой модели мобильной биомеотрической системы доступа. Данная модель получила название Mobile Biometric Lock (Mb-Lock). В рамках данной работы приведены основные результаты полученные в ходе ее разработки и апробации.

\clearpage